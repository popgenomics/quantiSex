\documentclass[25pt, a0paper, portrait, margin=0mm, innermargin=15mm,
     blockverticalspace=15mm, colspace=15mm, subcolspace=8mm]{tikzposter} 
\tikzposterlatexaffectionproofoff

% LATEX PACKAGES
% --------------

\usepackage{graphicx}  % package for inserting images, including .pdf
\usepackage{adjustbox} % package for cropping images
\usepackage[colorlinks=true, urlcolor=blue]{hyperref} % package for url and hyperlinks
\usepackage{wrapfig}
\usepackage[utf8]{inputenc}
\usepackage{wasysym}
\usepackage{fontawesome}
\usepackage{marvosym}


% TITLE, AUTHORS, INSTITUTE
% -------------------------

\title{\textbf{Sex allocation in hermaphroditic metapopulations}}
\author{Camille Roux\textsuperscript{2}, Charles Mullon\textsuperscript{1}, Samuel Neuenschwander\textsuperscript{1}, Jérôme Goudet\textsuperscript{1} and John R. Pannell\textsuperscript{1}}

\institute{\textsuperscript{1} DEE, University of Lausanne, Switzerland \\ 
\textsuperscript{2} UMR 8198 - Evo-Eco-Paleo, University of Lille, France \\ 
Website: \texttt{\href{https://github.com/popgenomics/quantiSex}{https://github.com/popgenomics/quantiSex}}}
%E-mail: \texttt{\href{mailto:john.pannell@unil.ch}{john.pannell@unil.ch}}}

% THEME SETTING
% -------------

%\usetheme{Rays}
\usetheme{Simple}
\useblockstyle{Minimal}
%\usecolorstyle[colorPalette=BlueGrayOrange]{Default}
%\useblockstyle{Default}
%\usebackgroundstyle{Default}
%\usetitlestyle{Wave}


% HEAD
% ----

\begin{document}
\maketitle
\begin{columns}

% ------------------------
% COLUMN 1 ---------------

\column{0.5}

% AIMS
% ----

\block{A. Sex allocation in \textbf{dioecious or gonochoristic} species}
{
	\begin{enumerate}
		\item \textbf{Sex allocation $\approx$ Proportion of {\faMars} and {\faVenus} produced by mothers} (weighted by the relative sex-specific cost of raising offspring). 
		\item \textbf{In panmictic populations $\Rightarrow$ sex-ratios are balanced.}

		\begin{center}
		\includegraphics[scale=0.75]{/home/croux/Documents/poster/jacques_monod_2018/selection_minority_sex_v2.pdf}
		\\Selection for the minority sex
		%\small{"Ancient coral reefs" by Heinrich Harder (1858-1935) - Wikimedia Commons}
		\end{center}
		
		\item \textbf{Non-random mating $\Rightarrow$ sex-ratios are biased}.\\
		Competitions among sons for mating ($=$\textbf{L}ocal \textbf{M}ate \textbf{C}ompetition) favours the selection for strategies that bias the sex-ratio towards the production of fewer \textbf{{\faMars}} and more \textbf{{\faVenus}}.
	
		\begin{center}
		\includegraphics[scale=1.2]{/home/croux/Documents/poster/jacques_monod_2018/figure_2_fig_v3.pdf}
		\end{center}

		\Large{Strong LMC $\Rightarrow$ bias toward daughters} \\
		\Large{\textcolor{green}{Decreased LMC $\Rightarrow$ production of sons closer to 50\%}} \\
			
	\end{enumerate}
}

\block[bodyoffsety=2cm,titleoffsety=2cm]{B. Sex allocation in demographically stable \textbf{hermaphrodite} ({\LARGE\Hermaphrodite}) populations}
{
	\begin{enumerate}
		\item \textbf{Sex allocation $=$ Relative investment made to {\faMars} \textit{versus} {\faVenus} functions by {\LARGE\Hermaphrodite} individuals.}
		\item \textbf{If random mating $+$ large dispersion} $\Rightarrow$ {\LARGE\Hermaphrodite} individuals favour \textbf{equal investment in both {\faMars} and {\faVenus} functions} (=50\% of {\faVenus} allocation).
		\item \textbf{If limited dispersal} $\Rightarrow$ Sib competition $\Rightarrow$ Selection for \textbf{increased investment} in the sex that shows the \textbf{smaller degree of competition} between siblings.

		\begin{center}
		\includegraphics[scale=1.2]{/home/croux/Documents/poster/jacques_monod_2018/selfing_herma_FIS_simulations.pdf}
%		Relation between selfing rate in {\LARGE\Hermaphrodite} and optimal {\faVenus$_{allocation}}
		\\Relation between selfing rate in {\LARGE\Hermaphrodite} and optimal \faVenus allocation.
		%\small{"Ancient coral reefs" by Heinrich Harder (1858-1935) - Wikimedia Commons}
		\end{center}
		A) \faVenus$_{allocation} = \frac{1+s}{2}$ (with symetrical cost and no inbreeding depression)\\
		B) $F_{IS} = \frac{s}{2-s}$\\
		C) \textbf{\faVenus$_{allocation}$ is difficult to measure directly from phenotypical traits}, but can be estimated by $F_{IS}$ from neutral molecular markers.\\
	\end{enumerate}
}

% DATA
% ----

%\block{Data}
%{
%	\begin{itemize}
%		\item Species distributional data from the
%		 \textbf{GASPAR database} (Kulbicki \textit{et al.} 2013, 
%		 Pravicini \textit{et al.} 2013), with 5331 species in 
%		 249 grid 5-degree cells that cover 99.76\% of 
%		 global coral reef area. 
%		\item Data on species-specific \textbf{vulnerability
%		to fishing and coral bleaching} 
%		(Cheung 	\textit{et al.} 2005, Graham \textit{et al.} 2011).
%		\item Data on five cell-specific 
%		characteristics (Paravicini \textit{et al.} 2014):
%	\end{itemize}
%	\begin{center}
%		\adjustbox{trim={.0\width} {.0\height} {0.0\width} {.35\height},clip}%
%		{\includegraphics[scale=1.4]{Figure_1.pdf}}	
%	\end{center}
%}

% METHODS
% -------

\block[bodyoffsety=4cm,titleoffsety=4cm]{C. Questions}
{
\textbf{\large \textcolor{green}{High levels of inbreeding can also emerge in outcrossers from metapopulation dynamics (local extinctions and recolonisations).}}

\begin{enumerate}
\item
\textbf{\large \textcolor{red}{How rapid must population turnover be to expect a strong biased sex allocation in a metapopulation of {\LARGE\Hermaphrodite}?}}
\item 
\textbf{\large \textcolor{red}{What index of inbreeding would be the best predictor of the sex allocation selected?}}
\end{enumerate}
}

% ------------------------
% COLUMN 2 ---------------

\column{0.5}

% RESULTS 
% -------

\block{D. Quantitative genetic simulations of {\LARGE\Hermaphrodite} metapopulations}
{
Model:
\begin{itemize}
  \item Multi-deme metapopulation made up of {\Large\Hermaphrodite}.
  \item Sex allocation is a function of the additive effects of alleles at a single locus subject to recurrent mutations that alter the allelic effects.
  \item Mating within demes and seed production depends on the sex allocations of individuals.	
  \item Demes are subject to recurrent stochastic extinction, following which their sites are recolonized through seed dispersal from the rest of the metapopulation.
  \item Population genetics statistics $F_{ST}$, $F_{IS}$, $G'_{ST}$ and $D_{Jost}$ were computed at 20 unlinked neutral loci.


   \begin{center}
   \includegraphics[scale=1.4]{/home/croux/Documents/poster/jacques_monod_2018/results_v2.pdf}
   \end{center}
%  \begin{center}
%  \adjustbox{trim={.0\width} {.52\height} {0.0\width} {.0\height},clip}%
%		{\includegraphics[scale=1.4]{Figure_3.pdf}}
%  \end{center}

   \item \faVenus$_{allocation}$ varies from 0.5 to 1 and $\nearrow$ if: migration $\searrow$ or extinction $\nearrow$ (\textbf{A}).
   \item Increased gene flow progessively $\searrow$ the equilibrium \faVenus$_{allocation}$ to values expected for single partial selfing populations (\textbf{B}).
\end{itemize}
}


\block[bodyoffsety=3cm,titleoffsety=3cm]{E. Predicting {\faVenus$_{allocation}$} from neutral molecular markers}
{
\begin{center}
   \includegraphics[scale=1.3]{/home/croux/Documents/poster/jacques_monod_2018/figure_stat_alloc_v2.pdf}
\end{center}

\begin{itemize}
  \item \textbf{No positive association between {\faVenus$_{allocation}$} and ($F_{IS}$} (Pearson's $R^{2}=0.0006$; $p-value=0.5494$).
  \item \textbf{{\faVenus$_{allocation}$} is most associated with $F_{ST}$} (Pearson's $R^{2}=0.84$; $p-value<2.2x10^{16}$).
  \item New indices of differentiation $G'_{ST}$ and $D_{Jost}$ $\searrow$ when migration $\nearrow$, but also when extinction $\nearrow$.
  \item Low $D_{Jost}$ can thus describe two opposite situations: \textbf{no population turnover} or \textbf{extreme population turnover}.
\end{itemize}
}


%\block{Results 2: Removing presences in site-species occurrence (SSO) matrix}
%{
%\begin{center}
%	\includegraphics[scale=1.5]{Figure_5.pdf}
%\end{center}
%
%}

% CONCLUSIONS
% -----------

\block[bodyoffsety=2cm,titleoffsety=2cm]{Conclusions}
{
%\begin{itemize}
\begin{enumerate}
\item Population \textbf{turnover should select for {\faVenus}-biased allocation} if migration is insufficiently strong to erase the genetic signatures of inbreeding brought about by colonisation.
\item In metapopulations: \textbf{$F_{ST}$ is a much better predictor of the sex allocation selected than $F_{IS}$}.
\end{enumerate}

%\end{itemize}

}

% REFERENCES
% ----------

\block[bodyoffsety=3cm,titleoffsety=3cm]{References}
{
\begin{small}

  \hangindent=2cm Düsing, K (1883) Die Factoren, welche die Sexualität entscheiden (Dissertation vorgelegt).

  \hangindent=2cm Fisher, R.A (1930) The Genetical Theory of Natural Selection (Clarendon Press, Oxford).

  \hangindent=2cm Hamilton, WD (1967) Extraordinary sex ratios. 
  \textit{Science}, \textbf{156} (3774): 477–88.

  \hangindent=2cm Charlesworth, D. and Charlesworth, B. (1981) Allocation of resources to male and female functions in hermaphrodites. \textit{Biological Journal of the Linnean Society}, \textbf{15} (1): 57-74.
  
  \hangindent=2cm Herre, E. A. (1985) Sex ratio adjustment in fig wasps. \textit{Science}, \textbf{228} (4701): 896-898.

  \hangindent=2cm David, P., Pujol, B., Viard, F., Castella, V., and Goudet, J. (2007) Reliable selfing rate estimates from imperfect population genetic data. \textit{Molecular ecology}, \textbf{16}(12): 2474-2487.

  \hangindent=2cm Whitlock, M. C. (2011). $G'_{ST}$ and $D_{Jost}$ do not replace $F_{ST}$. \textit{Molecular ecology}, \textbf{20}(6), 1083-1091.
\end{small}
}

% ACKNOWLEDGEMENTS
% ----------------
%\note[targetoffsetx=-8cm, targetoffsety=-9cm, width=25cm, innersep=1cm]
%{
%    \begin{wrapfigure}{r}{3cm}	
%    \vspace{-23pt}
%	\includegraphics[scale=3]{eu.jpg}
%	
%	\includegraphics[scale=3]{fp7.jpg}
%	\end{wrapfigure}
%	\textbf{Funding:} This research has received funding 
%	from the People Programme (Marie Curie Actions) of the EU 7th Framework
%	Programme (FP7/2007-2013), REA grant agreement no. 302868.
%    
%}


\end{columns}

% ----------------
\end{document}
\endinput
%%
%% End of file 
